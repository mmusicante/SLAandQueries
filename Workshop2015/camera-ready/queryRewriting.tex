%-----------------------------------------------------------------
%\section{Query Rewriting}
%\label{sec:queryRew}
%-----------------------------------------------------------------


Query rewriting is a well-known problem in the database domain that consists in transforming an abstract query into a set (or list) of lower-level queries that can be solved by  available databases.
%Query rewriting is guided by the schema of  abstract and concrete databases.
The answers to the lower level queries are combined in order to obtain the result for the higher-level query.
The query rewriting problem can be generalized to the case of services.
In this case, the query to be rewritten is seen as an \textit{abstract} service composition, to be expressed in terms of \textit{concrete} services.
Query rewriting techniques have been adapted to the context of data providing services~\cite{BBM10,ZLC11} as well as for general service compositions~\cite{CostaAMR13}. 
The case of more general services (\textit{i.e}, services that maintain and update information) is a generalization of the information-provision case.
Unlike the simpler case, where only the service interfaces need to be considered, the general case requires considering the functional and non-functional aspects of the query and available services.
%In this context, the \textit{Local as View} (\textit{LAV}) methods of query rewriting~\cite{Levy2000} are suitable.

In~\cite{CostaAMR13} the authors present an algorithm to automatically refine high-level specifications of service compositions into lower-level ones. 
The method is based on the MiniCon algorithm~\cite{PH01} for query rewriting.
%In the LAV approach, 
The rewriting process is guided by the specification of both the query to be rewritten and the available services.
%The specification of the composition to be produced as well as the specification of each available service are used in the case of general services.
%These specifications may detail both the functional and non-functional behaviour of each service, including SLA.
The approach consists in generating translations of an abstract query into \textit{several}  compositions over concrete  services. 
The solutions proposed are ranked and may be coded into concrete workflows as shown in the following sections.  
The next example shows the main features of the approach proposed in~\cite{CostaAMR13} which is extended to the case of SLA-guided service-based data integration. 

\begin{example}[Service Refinement by Rewriting]\label{Ex:rew1}
Let us consider the MOOC scenario, where people can participate on a content trading pool.
We suppose that this hypothetical MOOC has a large number of participants that sell their content  to  consumers. 
The MOOC has four information hubs described by their {\em agreed SLA}s, located in four different geographic locations, connected to other content providers.
These later may be  contacted directly, through the services of three different cloud providers that expose their {\em Cloud SLA}.

When an on-line consumer searches servers to buy/retrieve content, a composed  service is computed, in order to fetch each individual or hub service and to start  content retrieval procedures.
Depending on the location of the consumer and producers, different conditions and constraints may apply.
Also  cloud providers  publish the conditions for using their services.
These conditions and non-functional requirements (such as authentication or security requirements) should  be also expressed in the QoS user preferences and {\em agreed SLA}s.

In this context, the user  expresses a content order, her location and payment information. A composite  service is  generated to fulfill the order.
The generated service composition should consider the nearest providers, in accordance to the {\em derived SLA}.

In order to produce a personalized service composition for a user, the algorithm in~\cite{CostaAMR13} takes into account the specification of the composition. The specification of each available service is also considered. 
%~\hfill\openbox
\end{example}

It is worth to notice that the algorithm in~\cite{CostaAMR13} shows an exponential behavior, since it produces all combinations of concrete services that satisfy the abstract specification.
In order to solve this problem, we propose to use the \textit{agreed SLA} and user preferences to guide the choice of concrete services, to allow that the most preferred ones are combined first.
This will allow to produce the top \textit{k} rewritings first (in accordance to the user preferences), thus avoiding the cost of producing all the solutions.
%Given a set of services that can possibly be composed and the derived SLA, a service composition must be produced.
%Some of the inequations of the derived SLA should be included in the service compositions that answers the query.

The {\em agreed SLA} is included and merged with other SLA conditions that may be obtained by the rewriting process. Only those compositions that comply with the {\em derived SLA} will be presented as possible solutions to the user.
 

In the case of our query example the following composition can be used for answering it:

\begin{footnotesize}
\sf
\begin{tabbing}
 -~Q(myIPad\=dress, ``E.Dickinson'', 1MB, ``expert'', \$1, myCreditCard) $\equiv$ \\
 \>  myLoc = loc(myIPaddress), \\
 \>  estim$\_$cost(``E.Dickinson'', 1MB, cost, size, theirLocation), \\
 \>  query total cost + cost $\leq$ \$1,\\
 \>  availability $\geq$ 90$\%$, \\
 \>  freshness = any, \\
 \>  provenance = ``expert'', \\
 \>  duration = 7 days, \\
 \>  I/0 volume/month $\leq$ 8GB, \\
 \>  reputation level $\geq \lambda$ (the threshold) \\
 \>  storageSpace $\leq$ 1GB, \\
 \>  engage(``E.Dickinson'', size, myCreditCard).
 \end{tabbing} 
\end{footnotesize}

Our approach  generates a number $k$ of service compositions, combining as much as possible the services available such that the constraints of the {\em derived SLA} are verified. 
We are working on the modified version of the algorithm in~\cite{CostaAMR13} to take into account the different SLAs. 
 