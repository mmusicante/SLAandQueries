This paper introduces the challenge of integrating data from distributed data services deployed on different cloud providers guided by service level agreements (SLA) and user preferences statements. The data integration problem is stated as a continuous data provision problem that has an associated economic cost and that uses automatic learning techniques for ensuring different qualities of delivered data (fresh, precise, partial).

Current big data settings impose  to consider SLA and different data delivery models. We believe that given the volume and the complexity of query evaluation that includes steps that imply greedy computations, it is important to combine and revisit well-known solutions  adapted to these contexts. We are currently developing the strategies and algorithms sketched here applied to energy consumption applications and also to elections and political campaign data integration in order to guide decision making on campaign strategies.

Furthermore, the integration SLA generation step is outstanding study. In fact the integration SLA should indicate the negotiation result with the aim to reuse it in further integration. One of the possibilities when studying the integration is to accept the integration process in an incremental  manner so that the query will continue to be evaluated till satisfying the set of QoS and non functional constraints expressed by the user. To our knowledge, the incremental production of a solution is beyond  the scope of the current methods for rewriting service compositions and represents a challenge to the area.


% 
%\begin{example}[Incremental queries]\label{Ex:rew2}
%Let us return to the content  trade system of the MOOC example.
% Suppose that the user requires to retrieve a list of star providers  daily and weekly that can deliver ``\textit{5 Go}'' of expert content about Emily Dickinson poetry.
%In this case, each individual and hub database will be queried and the list will be produced by adding data obtained from them, until the 5Go  capacity is reached.
%Let us suppose that, in order to minimize the  communications between servers, the lists need to be produced incrementally.
%In this case, the composition produced by the service refinement may need to include an iteration (to aggregate partial results). 
%The data produced by the warehouse servers will be processed in batch and the process will end once the list reaches the desired capacity.
%
%
%~\hfill\openbox
%\end{example}
%
%\color{black}

