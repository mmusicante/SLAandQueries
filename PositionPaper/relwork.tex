
\begin{itemize}
\item Cloud computing represents a novel on-demand computing approach where resources are provided in compliance to a set of predefined non-functional properties specified and negotiated by means of Service Level Agreements (SLAs).  SLA currently exploited in the cloud allows service providers and cloud client to fix the resource level that should be used either by a service for the service provider or by the client in the case of service invokation. SLA could concern all types of the services on the cloud (IAAS, PAAS or SAAS). One of the weakness of SLA for the client is that they express low level clauses like storage amount use or virtualization level. In fact when a client launches a task on the cloud, she has unlikely Q \& S requirement rather than low level features. Recently we notice the existing of a lot of works with the objective to fill the gap.  Emeakaroha and al propose in \cite{5547150}  a cloud component that acts autonomously based on low level  monitored features after analysing them towards high level clauses expressed by the user. Moreover, another difficulty to enforce an SLA is to measure and identify, starting from a high level SLA clause, how could it be declined at different layers in the cloud. Therefore in \cite{Dastjerdi:2012:DOA:2275356.2275360} the authors describe a semanticSLA which can be understood by all parties including providers,
requestors, and monitoring services. One of the major objectives in the cloud is to anticipate SLA violation and to assess SLA failure cascading on violation detection. \cite{Dastjerdi:2012:DOA:2275356.2275360} proposes an SLA dependency modeling using Web Service Modeling Ontology (WSMO) to build a knowledge database.  \cite{5547150} proposes to anticipate failure by analysing the monitored feature and to act by anticipation. in \cite{5614035}, the authors propose LAYSI, a layered solution that minimizes user interactions with the system and prevents violations of agreed SLAs. 
On the other hand SLA contracts do not cover all client requirement. There is still a lack in some domain.....(to be completed)

 
  \item data integration (cf. travaux Gonzalez) 
\end{itemize}
