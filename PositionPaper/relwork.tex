
%\begin{itemize}
%\item Cloud computing represents a novel on-demand computing approach where resources are provided in compliance to a set of predefined non-functional properties specified and negotiated by means of Service Level Agreements (SLAs).  SLA currently exploited in the cloud allows service providers and cloud client to fix the resource level that should be used either by a service for the service provider or by the client in the case of service invokation. SLA could concern all types of the services on the cloud (IAAS, PAAS or SAAS). One of the weakness of SLA for the client is that they express low level clauses like storage amount use or virtualization level. In fact when a client launches a task on the cloud, she has unlikely Q \& S requirement rather than low level features. Recently we notice the existing of a lot of works with the objective to fill the gap.  Emeakaroha and al propose in \cite{5547150}  a cloud component that acts autonomously based on low level  monitored features after analysing them towards high level clauses expressed by the user. Moreover, another difficulty to enforce an SLA is to measure and identify, starting from a high level SLA clause, how could it be declined at different layers in the cloud. Therefore in \cite{Dastjerdi:2012:DOA:2275356.2275360} the authors describe a semanticSLA which can be understood by all parties including providers,
%requestors, and monitoring services. One of the major objectives in the cloud is to anticipate SLA violation and to assess SLA failure cascading on violation detection. \cite{Dastjerdi:2012:DOA:2275356.2275360} proposes an SLA dependency modeling using Web Service Modeling Ontology (WSMO) to build a knowledge database.  \cite{5547150} proposes to anticipate failure by analysing the monitored feature and to act by anticipation. in \cite{5614035}, the authors propose LAYSI, a layered solution that minimizes user interactions with the system and prevents violations of agreed SLAs. 
%On the other hand SLA contracts do not cover all client requirement. There is still a lack in some domain.....(to be completed)

%\item some 

Recent research works on SLA are devoted to the SLA extension to security concerns. Hale
and al. in \cite{6274042} propose an extension of the WSAgreement initially developed by the GRAAP
working group. The extension allows security constraints to be expressed over the service description
terms (SDTs) and the service level objectives (SLOs) of the SLA. Consequently, this leads to a certain kind of security expression interoperability and lets the client to be able to compare security level of different CSP. Luna and al
work \cite{LunaGarcia:2012:BCS:2381913.2381932} focuses on how to build a SEcSLA Template starting from gathering then categorizing a set of security statements using a semantic tool. The designed template is then used both to express user security requirements and CSP security provisions.

Another issue in the SLA management domain is how to let CSP offer fit the best user requirements. In fact, current SLA are only resource oriented and lacks in expressing client requirement in service quality and characteristics. Plenty of works tends to fill the gap. First by proposing predefined templates;. In \cite{Ortiz:2013:VPS:2486767.2486772}, the authors
  a set of templates to the user of cloud data services, each specifying the query type that can be executed with some tread-off in time and corresponding cost. 
  
  In \cite{LunaGarcia:2012:BCS:2381913.2381932,}, the authors propose to do a mapping of both user SecSLA requirements and CSP SecSLA provisions on a set of Quantitative Policy Tree(QPT), where atomic capabilities are mapped on leaves and intermediate represents coarse provisions with some logical operators (AND, OR). The mapping allows to give a quantitative benchmarking to each CSP offer with respect to user requirements that lets the user be able to make a choice. For more experienced users, \cite{6313668} proposes an extensible specification grammar that allows to express user and application-specific requirements in using a cloud data service. This allows to customize the SLA management at the cloud side. Emeakaroha and al propose in \cite{5547150}  a cloud component that acts autonomously based on low level monitored features after analysing them towards high level clauses expressed by the user. Moreover, another difficulty to enforce an SLA is to measure and identify, starting from a high level SLA clause, how could it be declined at different layers in the cloud. Therefore in  \cite{Dastjerdi:2012:DOA:2275356.2275360}  the authors describe a semantic- SLA which can be understood by all parties including providers,requestors, and monitoring services. 
  
Finally, some works tend to deal with SLA violation anticipation and SLA failure cascading on violation detection.  \cite{Dastjerdi:2012:DOA:2275356.2275360}  proposes an SLA dependency modeling using Web Service Modeling Ontology (WSMO) to build a knowledge database. \cite{5614035} proposes to anticipate failure by analysing the monitored feature and to act by anticipation. in
\cite{5614035}, the authors propose LAYSI, a layered solution that minimizes user interactions with the system and prevents violations of agreed SLAs.

 
 % \item data integration (cf. travaux Gonzalez) 
%\end{itemize}
