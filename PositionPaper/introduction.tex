The emergence of new architectures like the cloud opens new opportunities to data processing. 
The possibility of having unlimited access to cloud resources and the ``pay as U go'' model make it possible to change the hypothesis for processing big  data collections. 
Instead of designing processes and algorithms taking into consideration  limitations on resources availability, the cloud sets the focus on the economic cost implied of using resources and producing results by parallelizing their use owhile delivering data under subscription oriented cost models.
 
Integrating and processing heterogeneous data collections, calls for efficient methods for correlating, associating, filtering them taking into consideration their ``structural'' characteristics (due to the different data models) but also their quality, e.g., trust, freshness, provenance, partial or total consistency. 
Existing data integration techniques have to be revisited considering weakly curated and modeled data sets. This can be done according to quality of service requirements expressed by their consumers and Service Level Agreement (SLA) contracts exported by the cloud providers that host  these collections and deliver resources for executing the associated management processes.
% the economic cost derived of using computing, memory and storage resources for processing them.

%{\color{green} 
%Furthermore, security aspects are crucial in integrating big data. Indeed, the integration task must guarantee the integrity and privacy of data, the willingness of the clouds to  contribute to the security process through acceptable runtime environment conditions, and the adequacy of the proposed security levels to the services behind the integration operation.
%}

Our work addresses big data collections integration  in a multi-cloud hybrid context guided by user preferences statements and SLA contracts exported by different cloud providers. The objective is to propose an SLA guided continuous data provision and integration system exported as a DaaS by a cloud provider adapted to the vision of the economic model of the cloud such as accepting partial results delivered on demand or under predefined subscription models that can affect the quality of the results; accepting specific data duplication that can respect privacy but ensure data availability; accepting to launch a task that contributes to an integration on a first cloud whose SLA verifies a QoS requirement rather that a more powerful one.  

Therefore this paper presents an approach proposing strategies for computing integrated SLA’s according to agreed SLA’s exported by services  and adaptable query rewriting for integrating data sets  according to user preference statements.
This implies to consider several granularities of SLA: first, at the cloud level; the SLA ensured by providers regarding data; then at the service level, as unit for accessing and processing data, to be sure to fit particular service needs; and finally at the integration level i.e the possibility to process, correlate and integrate big data collections distributed across different cloud storage supports, providing different quality properties to data (trust, privacy, reliability, etc).
 
 
%Given the computational cost of a query evaluation and a user profile, our approach uses automatic learning techniques for generating knowledge out of every task and reducing query evaluation economic cost.


Accordingly, the remainder of this paper is organized as follows. Section \ref{sec:relWork} presents related works that address SLA modelling, integration and SLA guided data management processes. Section \ref{sec:incremental} gives an overview of our approach for integrating data sets provided by services (i.e., DaaS) by concilating SLA's provided by services and user's profiles expressing QoS preferences about the data they want to consume and the conditions in which they must be processed and delivered. 
%Section \ref{sec:incremental} introduces on demand and incremental data integration strategies. 
%Section \ref{sec:useCase} presents a use case for illustrating the interest and use of our approach. 
Finally \ref{sec:conclusions} concludes the papers and discusses future work.