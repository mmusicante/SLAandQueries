{\color{green}
Overview of our approach that will include:
- An SLA Model: including security issues

- A multi cloud environment representation

- On demand incremental data integration strategies
}

\begin{figure}
\caption{General architecture of an SLA guided  data integration system.\label{fig:arch}}
\end{figure}

Figure~\ref{fig:arch} shows the general architecture of an SLA guided data integration system that is supported by data services which are data providers deployed in a cloud and that provide agreed SLA’s. 
These descriptions are stored in a directory together with meta-data about the way queries are evaluated for producing results. 
The system uses this information  by query processing and monitoring modules for rewriting queries according to given quality of service (QoS) preferences expressed by a data consumer, for example a user.

\subsection{SLA model}
\label{sec:slaModel}

\begin{itemize}
\item Expression haut niveau du SLA en termes de préférences qui doit converger avec le SLA technique des services.
  \begin{itemize}
  \item (Souhait de temps de réponse, coût des services, espace de stockage,  
  \item Templates pour exprimer le SLA
  \item Intégration: modèle pivot de SLA
\end{itemize}

\item SLA violation contrôlée avec des mechanisms de monitoring.
\item Que ce que devient l’intégration de données par rapport au SLA
\item Création dynamique de SLA → niche de marché: étant donnée deux SLA fabriquer un SLA d’intégration
\end{itemize}


\subsection{Query Rewriting}
\label{sec:queryRew}

{\color{red}
Here Martin will explain the rewriting problem.
}

\subsection{Data and query models}
\label{sec:dqm}

TBD.
