{\color{green}
Overview of our approach that will include:
- An SLA Model: including security issues

- A multi cloud environment representation

- On demand incremental data integration strategies
}

\begin{figure*}
\caption{General architecture of an SLA guided  data integration system.\label{fig:arch}}
\end{figure*}

Figure~\ref{fig:arch} shows the general architecture of an SLA guided data integration system that is supported by data services which are data providers deployed in a cloud and that provide agreed SLA’s. 
These descriptions are stored in a directory together with meta-data about the way queries are evaluated for producing results. 
The system uses this information  by query processing and monitoring modules for rewriting queries according to given quality of service (QoS) preferences expressed by a data consumer, for example a user.

\subsection{SLA model}
\label{sec:slaModel}

\begin{itemize}
\item Expression haut niveau du SLA en termes de préférences qui doit converger avec le SLA technique des services.
  \begin{itemize}
  \item (Souhait de temps de réponse, coût des services, espace de stockage,  
  \item Templates pour exprimer le SLA
  \item Intégration: modèle pivot de SLA
\end{itemize}

\item SLA violation contrôlée avec des mechanisms de monitoring.
\item Que ce que devient l’intégration de données par rapport au SLA
\item Création dynamique de SLA → niche de marché: étant donnée deux SLA fabriquer un SLA d’intégration
\end{itemize}


In order to propose an SLA guided continuous data provision and integration, we need to think about possible steps from the request to the delivery of the result sets. Indeed, let consider a request R launched by a user who specifies a number of constraints on the environment execution. Executing this query requires first a semantic analysis which will subdivide R into a set of sub-queries, in such a way that each sub-query can be processed by a DataService deployed on the cloud. One may think to a first filter to remove the individual services which do not meet some or all of the constraints expressed by the user. This first stage can be defined as a vertical mapping SLAs given the high level SLA described by the user (i.e. macroscopic constraints: execution time, pay / no pay, data reliability, data source). The system should be able to find relevant service compositions that respond to the query and, when combined, meet the constraints imposed by the user (High level SLA).
To meet this objective, it is necessary to compare the SLA services to combine, in order to check if their joint use is compatible with the individual SLA. This step may lead either to the rejection of integration in case of total incompatibility, or to a negotiation between SLA which will lead us to the proposal for a negotiated SLA integration and thus the need for an adaptive Template.
The negotiation of this type of SLA depends strongly on the request sent and the services deployed at the arrival time of the application on the cloud. This negotiation can be expensive and may not scale well. It is therefore crucial to provide proactive mechanisms for optimizing the production of such SLA. We believe that the optimization of this process can occur at two levels, firstly at the level of SLA previously traded, and secondly at the level of partial or total results. Indeed, queries requesting the same services compositions will have clauses in their SLAs that are more conditions of use of the infrastructure (ie not touching the data). For two different queries, they will be negotiated in the same way. These previously negotiated SLAs are reusable.
In a second time, we think optimizing this process on the data storage mechanisms to cache intermediate results, individually or in partial or complete combination depending on the terms of SLA services (data access, intermediate storage capacity , cost of storage , etc ... ).
Given this proposal, we identify several issues:
- Level modeling would require a model that allows the representation of SLA integration.
- There should also be a template for representing the requirements expressed by the user
- A mining component to identify, from the requirements expressed in the template by the user, and before the analysis of the application, the candidate integration SLA to use or adapt according to the request. This implies mapping between property and expressed clauses being.

\subsection{Query Rewriting}
\label{sec:queryRew}

{\color{red}
Here Martin will explain the rewriting problem.
}

\subsection{Data and query models}
\label{sec:dqm}

TBD.
