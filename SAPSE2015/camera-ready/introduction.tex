%\color{red}
%


\color{black}
The advent of cloud computing has imposed a new   resource consuming model that focuses on
 the \textit{technical and economic} conditions to be fulfilled in order to access potentially unlimited resources. Integrating and processing heterogeneous data collections, calls for efficient methods for correlating, associating, and filtering them considering their variety (i.e., different formats and data models) and   quality, e.g., trust, freshness, provenance, partial or total consistency. 
Existing data integration techniques have to be revisited considering weakly curated and modeled data sets. This can be done according to (i) quality of service (QoS) requirements expressed by their consumers and Service Level Agreement (SLA) contracts exported by data services; (ii)  cloud providers that host  these collections and deliver resources for executing  data processing and integration processes.
In fact,  the interest of SLA has been demonstrated  in data analysis but has not been yet widely considered for integrating data. 
We believe in the benefits of SLA-based data integration as an approach for better meeting  user  requirements related to the conditions in which data is delivered and integrated, and on the quality of the data provided by services.
To do so, several granularities of SLA must be considered: first, at the cloud level: the SLA ensured by providers regarding data; then at the service level, as unit for accessing and processing data, to be sure to fulfill  specific  needs; and finally at the integration level i.e the possibility to process, correlate and integrate big data collections distributed across different cloud storage supports, providing different quality properties for data (trust, privacy, reliability, etc).

 

To better understand our problem, consider a massive open online course system (MOOC) where users produce and consume content according to the geographic area and the expertise of participants. 
Producers are characterized according to their location, the type and topic of  content  they can provide, the access conditions (e.g. cost, inscription, or exchange unit), and the time window in which they can produce content. 
Consumers are characterized by their location, their interests  during a certain interval of time, the maximum total cost they are ready to pay consuming content, or the resources they are ready to provide in order to get the service, and QoS requirements such as availability and how critical it is to consume a given type of content. Both producers and consumers have subscriptions to different cloud providers for dealing with content storage, processing and exchange.
The MOOC  aims at being privacy respectful of the producers and consumers participating in courses.
 Providers  and consumers can ask to minimize the transfer of personal data  when they share/consume content.  According to the level of trust associated to data providers, the MOOC can use privacy preserving techniques to let users share content anonymously  (Constraint1). Furthermore,  data providers can  also wish to give restricted data access credentials w.r.t. to their  trust level, when their data are used  within  an integration process (Constraint 2).  
 

  From this scenario we identify a set of issues that have to be addressed. First, how can the user efficiently obtain  results for her queries such that they meet her QoS requirements while respecting her subscribed contracts with the involved cloud provider(s)?  The user knows the low-level clauses of her SLA contract but she does not have an idea about the resources required to satisfy her requirements (SLA exported by services, example of Constraint 1).  Second, data services contracts should not be neglected in an integration process (Constraint 2).   As data services are deployed in a multi-cloud context, there is a need to establish the rules to determine how integration can be done and in which conditions. Intuitively, integration can be  done enforcing all specified conditions; but it is also possible to  expect situations where conditions can only be verified partially. Furthermore,  matching data providers with requests and QoS preferences with SLA's can be computationally costly. For this reason  the results of  such a  complex process  should be capitalized for further integration requests. How can this be done? 
  
  The first step is to propose an architecture that integrates all services participating in a data integration process that copes QoS and SLA's of the services that participate in this process for computing a given query. 
  We have proposed such architecture in~\cite{BennaniGMV14}. Based on this architecture it is necessary to revisit  the integration phases  specification to include QoS and SLA. The objective being to  propose a new way of dealing with a data production and consumption process, considering SLA formalism and clauses -especially security concerns-  and heterogeneity among clouds. Therefore, this paper introduces a preliminary solution addressing SLA-guided data integration focusing  on query rewriting addressing security issues. We assume that data services publish Agreed-SLA that describe their  static and dynamic QoS measures in specific registries.  Thanks to an extended query rewriting process, a new kind of SLA is derived to match user preferences with low level agreed SLAs exported by services. The derived SLA allows to choose  services (w.r.t user requirements) that can contribute for answering a query. Once data services have been chosen, the integration process will be guided by an integration SLA, negotiated according to user requirements and inter-cloud contracts.

 
This paper is organized as follows. Section \ref{sec:relWork} presents related works that address SLA modeling, integration and SLA guided data management processes in cloud environments. Section \ref{sec:incremental} gives an overview of our approach and highlights the need of new kind of SLA to conduct the integration process. Section \ref{sec:derivedSla}, \ref{sec:queryRew} and \ref{sec:queryProcessOpt} give an idea of the main steps of our proposal.  Finally section  \ref{sec:conclusions} concludes the paper and discusses future work.









 